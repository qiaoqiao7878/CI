\documentclass[a4paper,12pt]{article}
\usepackage[utf8]{inputenc}
\usepackage[margin=3cm]{geometry}
\usepackage{amsmath}
\usepackage{amssymb}
\usepackage{amsthm}
\usepackage{fancyhdr}
\usepackage{seminar}
\usepackage{graphicx}
\usepackage{subfigure}
\usepackage{float}
\usepackage{hyperref}
\pagestyle{fancy}


%You can add theorem-like environments (e.g. remark, definition, ...) if you want
\newtheorem{theorem}{Theorem}

\title{Assignment 1: Neural Networks} % Replace with your title
\author{Jonas Berger, Qiao Qiao} % Replace with your name
\institute{Technische Universit\"{a}t M\"{u}nchen} % Replace with the department you belong to

\makeatletter
\let\runauthor\@author
\let\runtitle\@title
\makeatother
\lhead{\runauthor}
\rhead{\runtitle}


\begin{document}

\maketitle

\section{Introduction}
...



\section{Task 1: Function Approximation}

\subsection{Data Preparation}

\begin{itemize}

 \item Plot the parametric surface and contour of the target function

\end{itemize}

\subsection{Network Design}

\begin{itemize}

	\item Plot the contours of the NN with 2, 8 and 50 hidden neurons.

	\item Which is the best hidden neurons number among them? Why?

\end{itemize}

\subsection{Network Training}

\begin{itemize}
	
	\item Compare the training performance of 4 different training functions in three aspects, epochs, training time and correlation and fill in the table below. State the features of four training algorithms according to the table.\\

	\item Plot the contours of the NN with four different training algorithms.

\end{itemize}

\begin{center}
	\begin{tabular}{|c | c | c | c|} 
		\hline
		training function & epochs & total time(sec) & Correlation \\ [0.5ex] 
		\hline
		trainbfg & ... & ...  & ... \\ 
		\hline
		traingdm & ... & ... & ... \\
		\hline
		traingd & ... & ... & ... \\
		\hline
		trainlm & ... & ... & ... \\
		\hline
	\end{tabular}
\end{center}

\subsection{Network Testing}

\begin{itemize}
	
	\item Plot the graphical output provided by \texttt{postreg}.

\end{itemize}


\section{Task 2: System identification}

\begin{itemize}

	\item Generate performance and training state plots for each of the cases.


	\item Run the control system with each of the 3 cases and generate the reference tracking plots for them (X-Y plot in the default setup).

	\item Do brief comparisons on the performance of each of the 3 cases based on the generated plots.

	\item How relevant is the number of training epochs in this context?

	\item Shortly justify which of the 3 sets of identification data you would pick to use.

	\begin{equation}
	SE(3) = \left\{
	\begin{pmatrix}
	R & \mathbf{t} \\ \mathbf{0} & 1
	\end{pmatrix}
	:
	R^\top R = R R^\top = I_3, \mathbf{t}\in\mathbb{R}^3
	\right\}
	\end{equation}
	
\end{itemize}

\end{document}